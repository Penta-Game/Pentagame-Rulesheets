% Hi Firas, you should in theory be able to just replace the English text with Arabic. I am curious to see how that will work!

\selectlanguage{arabic}
\renewcommand{\headline}{Penta$\cdot$game - العربيةh}
\renewcommand{\tocent}{العربية}
\renewcommand{\translator}{Translator Name}
\renewcommand{\general}{
%% example text, proof of concept - don't know what this means, just delete it...

لعبة يمكن تفسيرها في دقيقة 
A game that can be explained in a minute yet remains fascinating
for years.

\textLR{Two players play in just 20 minutes. Three or four players may take up to 90. }

There are no dice involved. Suitable for all  from age~5. A game by Jan \noun{Suchanek.} 

In the box: 4$\times$5 coloured figures, 5 black and 5 grey blocks, the board.
}

\renewcommand{\choosext}{Choose your figures}
\renewcommand{\choosex}{
Every player has five figures. There is one team of figures per player in the box. 

You have a blue, a red, a white, a green and a yellow figure. 

They start at the five corners of the board matching their colour.

They want to reach the big stops in the middle, where they move out.
}
\renewcommand{\setupt}{Setup}
\renewcommand{\setup}{
Put your figures on the big corners at the ring matching their body colours: 
your white figure on the white corner at the ring, your blue figure on the blue corner at the ring, etc.

Put the black blocks on the five crossings in the middle of the board. 

Park the grey blocks in the middle for later. 
}

\renewcommand{\objectivet}{Objective}
\renewcommand{\objective}{
White figures want to go to the white crossing in the centre, blue
ones to the blue crossing, etc. The destination is always the big
coloured stop in the middle pentagon opposite the starting point.

Be the first to move \emph{three} pieces to their destinations to win.
}

\renewcommand{\rulest}{The Rules}
\renewcommand{\rules}{
Move any of your figures on the star or the ring in any direction as far as you please. 

You can turn at any free corner or crossing without stopping.

Never jump\textemdash neither over blocks, nor over any figures. 

\myskip

But you may move \emph{onto} a stop that is occupied:

\myskip

If you then beat a black block, place it on a free stop of your choice.

If you then move onto a stop with another figure, swap position with it.
 
\myskip

You may swap the positions of two of your own figures. 

If you move onto a stop with multiple figures, choose one to swap with.

\myskip 
 
Do not make the very same move twice in immediate succession.  

\myskip 

A figure that has reached its destination is removed. Put it into the centre of the board. Then take one of the grey blocks and put it on a free stop of your choice.

If you beat such a grey block, take it from the board again.

\myskip 

If another player has  moved you to your goal, you \emph{must }move out once it is your turn.

\myskip

Whoever moves \emph{three }of his figures out first wins.%
}