\selectlanguage{finnish}
\renewcommand{\headline}{\section*{{\LARGE{}Penta$\cdot$game} - Suomi}}
\renewcommand{\tocent}{Suomi}
\renewcommand{\translator}{ihmis-suski, bergie}
\renewcommand{\general}{
  Peli, jonka voi selittää minuutissa, mutta joka säilyy mielenkiintoisena vuosia.

  Kaksinpelin pelaa noin 20 minuutissa. Kolmen tai neljän pelaajan peli saattaa kestää jopa 90 minuuttia. 

  Pelissä ei käytetä noppaa. Peli sopii kaikille 5 vuotiaasta ylöspäin.

Sisältö: 
  5 mustaa nappulaa
  5 harmaata nappulaa
  20 värillistä nappulaa, viittä eri väriä
  pelilauta
}

\renewcommand{\choosext}{Pelinappulan valinta}
\renewcommand{\choosex}{
Jokaisella pelaajalla on viisi saman muotoista mutta eriväristä nappulaa: sininen, punainen, valkoinen, vihreä, ja keltainen.

Nappulat asetetaan pelilaudan viiteen eriväriseen kulmaan nappulan värin mukaan.

Pelaajien tavoitteena on saada nappulansa niiden väriä vastaaviin ympyröihin laudan sisäkehällä.

}

\renewcommand{\setupt}{Pelin valmistelu}
\renewcommand{\setup}{
Aseta nappulasi laudan ulkokehällä oleviin isoihin värillisiin ympyröihin siten, että ympyrän ja nappulan väri ovat samoja: Sininen nappula siniseen ympyrään, valkoinen nappula valkoiseen ympyrään jne.

Aseta mustat nappulat laudan keskellä oleviin värillisiin ympyröihin.

Aseta harmaat nappulat pelilaudan keskelle myöhempää käyttöä varten.
}

\renewcommand{\objectivet}{Tavoite}
\renewcommand{\objective}{
  Pelin tavoitteena on siirtää nappulat ulkokehän värillisestä ympyrästä sisäkehän saman väriseen ympyrään. Siniset nappulat ulkokehän sinisestä ympyrästä sisäkehän siniseen ympyrään, valkoiset valkoiseen jne.

Ensimmäinen pelaaja, joka saa siirrettyä \emph{kolme} nappulaansa sisäkehän nappulan väriä vastaavaan ympyrään, voittaa.
}

\renewcommand{\rulest}{Säännöt}
\renewcommand{\rules}{

Nappuloita voi liikuttaa tähden sakaroita tai ulkokehän ympyrää pitkin mihin suuntaan tahansa, kuinka pitkälle tahansa.

Nappula voi kääntyä missä tahansa kulmassa pysähtymättä.

Nappuloita ei voi koskaan liikuttaa harmaan tai mustan esteen, tai minkä tahansa värillisen nappulan yli.

\medskip

Nappula voi pysähtyä ruutuun joka on jo varattu.

\medskip

Jos ruudussa johon nappula pysähtyi on musta pelinappula, pelaaja voi siirtää mustan nappulan vapaasti valitsemaansa ruutuun.

Jos ruudussa johon nappula pysähtyi on toinen värillinen pelinappula, tulee nappuloiden paikaa vaihtaa päittäin.
 
\medskip

Pelaaja voi vaihtaa kahden oman pelinappulansa paikkaa päittäin.

Jos ruudussa johon nappula pysähtyi on useampi pelinappula, pelaaja valitsee minkä niistä kanssa nappula vaihtaa paikkaa.
 
\medskip 
 
Samaa siirtoa ei voi tehdä kahdesti peräkkäin (esim. vaihtaa nappuloiden paikkoja päittäin).

\medskip 

Nappulan päästessä maaliin, se siirretään pelilaudan keskelle. Tällöin nappulansa maaliin saanut pelaaja ottaa yhden harmaan nappulan (jos niitä on vielä jäljellä) ja siirtää sen vapaasti valitsemalleen tyhjään ruutuun.

Jos nappula pysähtyi ruutuun jossa on harmaa pelinappula, otetaan se pelilaudalta ja siirretään se odottamaan pelilaudan keskelle.

\medskip

Pelaaja joka saa ensimmäisenä nappuloistaan \emph{kolme} pelilaudan keskelle voittaa.

\medskip

Peli loppuu kun kaikki kierroksen vuorot on pelattu.

Liikaa puhuva pelaaja diskataan.
}
