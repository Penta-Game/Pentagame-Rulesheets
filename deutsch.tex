\selectlanguage{ngerman}
\renewcommand{\headline}{Penta$\cdot$spiel - Deutsch}
\renewcommand{\tocent}{Deutsch}
\renewcommand{\translator}{J.S.}
\renewcommand{\general}{
\lettrine[lines=3]{E}{in} Spiel, das man in einer Minute begreift, das aber jahrelang fasziniert. 
Zu zweit dauert die Partie nur 20 Minuten. Drei oder vier Spieler spielen 40-90 Minuten.
Es gibt keine Würfel. Geeignet für Spieler ab 5 Jahren.
In der Schachtel: 4$\times$5 farbige Figuren, 5 schwarze und 5 graue Blöcke und das Spielbrett.
}

\renewcommand{\choosext}{Figuren}
\renewcommand{\choosex}{
Jeder Spieler führt eine Mannschaft von fünf Figuren einer Form.
Jeder sucht sich eine Mann\-schaft aus.

In jeder solchen Mannschaft ist   eine blaue, eine rote, eine weiße, eine grüne und eine gelbe Figur.

Diese laufen alle von den fünf Ecken zu den fünf Kreuzungen, wo sie herausziehen.
}

\renewcommand{\setupt}{Aufbau}
\renewcommand{\setup}{
Stellt die Figuren geordnet nach Farben auf die fünf Ecken auf dem Kreis: die weißen auf das weiße Eckfeld, die blauen auf das blaue usw.

Je ein schwarzer Block kommt auf die Kreuzungen in der Mitte.

Die grauen Blöcke bleiben erst mal im Zentrum.
}

\renewcommand{\objectivet}{Ziel des Spiels}
\renewcommand{\objective}{
Weiße Figuren möchten zu der weißen Kreuzung, blaue zur blauen usw. 
Vom Rand her ist das Ziel immer die Kreuzung gleicher Farbe gegenüber.

Wer zuerst \emph{drei} seiner Figuren auf ihre jeweiligen Ziele zieht, gewinnt.
}

\renewcommand{\rulest}{Spielregeln}
\renewcommand{\rules}{
Ziehe eine deiner Figuren auf Stern oder Kreis in beliebiger Richtung beliebig weit.

Du kannst dabei an jeder freien Kreuzung ohne anzuhalten abbiegen.

Du darfst ziehen, aber nicht springen! Weder über Blöcke noch über andere Figuren!

\myskip

Jedoch kannst du auf ein Feld ziehen, das besetzt ist, und schlagen:

\myskip

Schlägst du einen schwarzen Block, so setze ihn auf ein beliebiges freies Feld. 

Schlägst du eine andere Spielfigur, so tauschen deine und diese Figur ihre Positionen.

\myskip

Auf diese Weise kann man zwei seiner eigenen Figuren vertauschen.

Ziehst du auf ein Feld, auf dem noch mehrere Figuren stehen, musst du mit \emph{einer }von ihnen tauschen.


 
\myskip 
 
Man darf nicht denselben Zug zweimal machen. 

\myskip 

Hat eine Figur ihr Ziel erreicht, so zieht sie raus. Sie kommt ins Zentrum. Dafür darf man von dort einen grauen Block ins Brett auf ein freies Feld setzen.

Schlägt man so einen grauen Block, so kommt er wieder raus.

\myskip 

Wirst du von einem anderen Spieler auf dein Ziel gebracht, so \emph{musst }du raus ziehen, sobald du dran bist.

\myskip

Die letzte Runde wird zu Ende gespielt.

% Wer zuerst  \emph{drei }seiner Figuren herauszieht, gewinnt.%
}