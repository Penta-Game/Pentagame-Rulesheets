
\renewcommand{\headline}{Penta$\cdot$juego - Español}	
\renewcommand{\tocent}{Español}	
\renewcommand{\translator}{Juan José Bravo}	
\renewcommand{\general}{	
Un juego que se explica en un minuto pero que sigue fascinándote durante años.	
	
	
Una partida entre dos jugadores puede llevarte 20 minutos. Tres o cuatro jugadores pueden emplear hasta los 90 minutos.	
	
	
Requisitos: Se juega sin dados. A partir de cinco años.	
Realizado por  Jan \noun{Suchanek.}	
	
Contenido de la caja: 4 grupos de cinco peones de colores, 5 negros, 5 grises y un tablero.	
	
}		
	
\renewcommand{\choosext}{Elije tus peones}	
\renewcommand{\choosex}{	
Cada jugador dispone de cinco peones. Un equipo esta compuesta por cinco peones y hay cinco  por jugador en la caja. 	
%Hay un equipo con el  pelo plateado,un con pelo negro, otro rubio y otr sin pelo.	
	
%Los jugadores deben llevar el equipo que mas se les parezca. 	
	
Cada equipo esta compuesto por cinco peones; un peón azul, uno rojo, uno blanco, uno verde  y uno amarillo.	
	
Disponed vuestros peones en los círculos exteriores de los colores correspondientes:
Ellos tendrán que llegar a la meta situada en el medio del tablero. 	
}	

\renewcommand{\setupt}{Instalación}	
\renewcommand{\setup}{	
Disponed vuestros peones sobre los círculos exteriores en sus colores correspondientes: 
el peón blanco en el circulo exterior blanco, el azul en circulo exterior azul, etc.	
	
Dispón los peones negros en los cinco círculos interiores, en el medio del tablero.
Coloca los peones grises en el centro del tablero.	
}	
	
\renewcommand{\objectivet}{Objetivo}	
\renewcommand{\objective}{	
Los peones blancos deben alcanzar los círculos interiores blancos, los peones azules los círculos interiores azules, etc.	
El destino es siempre el gran circulo central de color en el medio del pentagama opuesto al punto de partida.	
	
	
Se el primero en colocar tres de tus cinco peones en los círculos de colores correspondientes para ganar.
}	
	
\renewcommand{\rulest}{Reglas del juego}	
\renewcommand{\rules}{	
Mueve el peón de tu elección sobre el tablero de juego, donde que quieras, sin importar la dirección, tan lejos como quieras.	
	
Tu puedes utilizar todos los caminos posibles.	
	
No podrás saltar sobre círculos ocupados.	
	
	
Pero podrás colocarte en un circulo  ocupado:	
	
	
Si caes en   un circulo ocupado por un peón negro, tomas este  y lo colocas en cualquier circulo libre, de tu elección.	
	
Si caes en un circulo ocupado por otro peón, cambias tu lugar anterior por el.	
	
Tu puedes cambiar  de posición entre dos peones de tu equipo. 	
	
Si caes en una casilla ocupada por varios peones, puedes elegir uno de ellos y cambiar de lugar con el.	
	
	
No puedes hacer el mismo movimiento dos veces seguidas.	
	
Cuando un peón llega a  la meta, es retirado del juego, se le coloca en el centro del tablero y se toma un peón gris que podrás colocar en cualquier circulo libre de tu elección.
	
Si caes en un lugar ocupado por un peón gris, este peón gris  sera retirado del juego.	
	

Si otro jugador te ha movido hacia tu meta, debes irte una vez que sea tu turno.
	
	
El primer jugador que consiga retirar tres peones ganará la partida.%
}

