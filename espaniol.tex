
\renewcommand{\headline}{Penta$\cdot$juego - Español}	
\renewcommand{\tocent}{Español}	
\renewcommand{\translator}{Juan José Bravo / Carlos Ayon}	
\renewcommand{\general}{	
Un juego que se explica en un minuto, pero que sigue fascinándote durante años.	
	
Una partida entre dos jugadores puede llevarte 20 minutos. Tres o cuatro jugadores pueden tardar hasta 90 minutos.	
	
Se juega sin dados. Apto para todos, a partir de los cinco años.	
Realizado por  Jan \noun{Suchanek.}	
	
En la caja: 4 x 5 figuras de colores, 5 bloques negros, 5 grises y un tablero.	
	
}		
	
\renewcommand{\choosext}{Elije tus figuras}	
\renewcommand{\choosex}{	
En la caja hay un grupo de figuras por jugador.Cada equipo esta compuesto por cinco figuras; una azul, una roja, una blanca, una verde  y una amarilla.	
Las figuras se colocan en los círculos exteriores de los colores correspondientes:
Ellas tendrán que llegar a la meta situada en el medio del tablero. 	
}	

\renewcommand{\setupt}{Instalación}	
\renewcommand{\setup}{	
Disponed vuestras figuras sobre los círculos exteriores en sus colores correspondientes: 
la figura blanca en el circulo exterior blanco, la azul en circulo exterior azul, etc.	
	
Dispón los bloques negros en los cinco círculos interiores en el medio del tablero.
Coloca los bloques grises en el centro del tablero.	
}	
	
\renewcommand{\objectivet}{Objetivo}	
\renewcommand{\objective}{	
Las figuras blancas deben alcanzar los círculos interiores blancos, las figuras azules los círculos interiores azules, etc.	
El destino es siempre el gran circulo central, del color correspondiente a la figura, en el medio del pentagono opuesto al punto de partida.	
	
Se el primero en colocar \emph{tres} de tus cinco peones en los círculos de colores correspondientes para ganar.
}	
	
\renewcommand{\rulest}{Reglas del juego}	
\renewcommand{\rules}{	
Mueve la figura de tu elección sobre el tablero de juego sin importar la dirección, tan lejos como quieras.	
	
Puedes girar en cualquier esquina o interseccion libres, sin detenerte.	
	
No podrás saltar sobre ningun bloque o figura, pero podrás colocarte en un circulo  ocupado:	
	
\myskip	
	
Si caes en un circulo ocupado por un bloque negro, tomalo y colocalo en cualquier espacio libre de tu elección.	
	
Si caes en un circulo ocupado por otra figura, intercambia tu lugar anterior con ella.	
	
	
Puedes cambiar de posición entre dos peones de tu propio equipo. 	


Si caes en una casilla ocupada por varias figuras, puedes elegir una de ellas e intercambiar de lugar.	

\myskip	
	
No puedes hacer el mismo movimiento dos veces seguidas.	

\myskip
	
Cuando una figura llega a su meta, es retirada del juego. Se le coloca en el centro del tablero y se toma un peón gris que podrás colocar en cualquier circulo libre de tu elección.
	
\myskip		
	
Si caes en un lugar ocupado por un bloque gris, este sera retirado del juego nuevamente.	

\myskip	

Si otro jugador te ha movido hacia tu meta, debes avanzar a ella una vez que sea tu turno.

\myskip	
	
El primer jugador que consiga retirar tres de sus figuras ganará la partida.%
}

