\titlespacing*{\section}
{0pt}{1ex plus 1ex minus .2ex}{-2.3ex plus .2ex}
\newcommand{\skipper}{\medskip\hrulefill\section{}}

\tikzset{state/.style={circle, draw, inner sep=3pt, opacity=1, fill=black, minimum size=0.2pt}}
\tikzset{empty/.style={circle, draw, inner sep=3pt, opacity=0, fill=white, minimum size=0.2pt}}
    \tikzset{cross/.style={cross out, draw=black, minimum size=2*(#1-\pgflinewidth), inner sep=2pt, outer sep=2pt},
    %default radius will be 1pt. 
    cross/.default={1pt}}  
  
\newpage



    \centering
    
    \skipper
    
    \begin{tikzpicture}[scale=0.65]

    \scriptsize
    \begin{scope}[xscale=-1]
    \def \n {5} \def \radius {3cm} \def \raz {1.4cm}
    \foreach \s in {1,...,\n} { 
    \draw[dashed, >=latex] ({360/\n * (\s - 1)}:\radius)      arc ({360/\n * (\s - 1)}:{360/\n * (\s)}:\radius);   
    \draw[dotted, >=latex] ({360/\n * (\s - 2)+90}:\radius)      -- ({360/\n * (\s)+90}:\radius); 
    } 
    
    \node[state, fill=white] (1) at ({360/\n * (1 - 1)+90}:\radius) {};  
    \node[state, fill=white] (1a) at ({360/\n * (1 - 1)+90}:-\radius/2.618) {}; 
    
    \node[state, fill=blue] (2) at ({360/\n * (2 - 1)+90}:\radius){};   
    \node[state, fill=blue] (2a) at ({360/\n * (2 - 1)+90}:-\radius/2.618) {}; 
    
    \node[state, fill=red] (3) at ({360/\n * (3 - 1)+90}:\radius) {};  
    \node[state, fill=red] (3a) at ({360/\n * (3 - 1)+90}:-\radius/2.618) {}; 
    
    \node[state, fill=yellow] (4) at ({360/\n * (4 - 1)+90}:\radius) {}; 
    \node[state, fill=yellow] (4a) at ({360/\n * (4 - 1)+90}:-\radius/2.618) {}; 
    
    \node[state, fill=green] (5) at ({360/\n * (5 - 1)+90}:\radius) {};   
    \node[state, fill=green] (5a) at ({360/\n * (5 - 1)+90}:-\radius/2.618) {}; 
    
    
    \draw[>=triangle 45, line width=1.5pt, ->] (1) -- (1a) ;
    \end{scope}
    \end{tikzpicture}
    %\caption{Objective}
    
    You have five pieces.
    
    Bring them to the their \textbf{goals opposite their origins:}
    
    White to white, blue to blue etc.
    
    \textbf{Three out wins.}

% \begin{multicols}{2}
% \raggedcolumns



 
  
\skipper 

    \begin{tikzpicture}[scale=0.65]
    \scriptsize
    \begin{scope}[xscale=-1]
    \def \n {5} \def \radius {3cm} \def \raz {1.4cm}
    \foreach \s in {1,...,\n} { 
    \draw[dashed, >=latex] ({360/\n * (\s - 1)}:\radius)      arc ({360/\n * (\s - 1)}:{360/\n * (\s)}:\radius);   
    \draw[dotted, >=latex] ({360/\n * (\s - 2)+90}:\radius)      -- ({360/\n * (\s)+90}:\radius); 
    } 
    %\draw[dashed] ({360/\n * (1 - 1)+95}:\radius)      arc ({360/\n * (1 - 1)+95}:{360/\n * (1)+85}:\radius)  node[midway, rotate=-18] {$\times$};
    \draw[>=triangle 45, line width=1.5pt, ->] ({360/\n * (1-1)+95}:\radius)  arc ({360/\n * (1-1)+95}:{360/\n * (2-1)+80}:\radius) node[midway,left]{};
    
    \draw[>=triangle 45, line width=1.5pt, ->] ({360/\n * (1-1)+95}:\radius)  arc ({360/\n * (1-1)+95}:{360/\n * (-1)+100}:\radius) node[midway,left]{};
    
    \draw[>=triangle 45, line width=1.5pt, shorten >=1.5ex, ->] (1) -- (4a);
    
    \draw[>=triangle 45, line width=1.5pt, shorten >=1.5ex, ->] (1) -- (3a);
    
    
    \node[state, fill=white] (1) at ({360/\n * (1 - 1)+90}:\radius) {};  
    \node[state, fill=white] (1a) at ({360/\n * (1 - 1)+90}:-\radius/2.618) {}; 
    
    \node[state, fill=blue] (2) at ({360/\n * (2 - 1)+90}:\radius){};   
    \node[state, fill=blue] (2a) at ({360/\n * (2 - 1)+90}:-\radius/2.618) {}; 
    
    \node[state, fill=red] (3) at ({360/\n * (3 - 1)+90}:\radius) {};  
    \node[state, fill=red] (3a) at ({360/\n * (3 - 1)+90}:-\radius/2.618) {}; 
    
    \node[state, fill=yellow] (4) at ({360/\n * (4 - 1)+90}:\radius) {}; 
    \node[state, fill=yellow] (4a) at ({360/\n * (4 - 1)+90}:-\radius/2.618) {}; 
    
    \node[state, fill=green] (5) at ({360/\n * (5 - 1)+90}:\radius) {};   
    \node[state, fill=green] (5a) at ({360/\n * (5 - 1)+90}:-\radius/2.618) {}; 
    
    \end{scope}
    \end{tikzpicture}
    %\caption{Simple move on ring}
    
    You can move \textbf{in any direction,} on the ring and on the star.
    
   \skipper

  

    %%%% EXAMPLE MOVES
    \centering
    \begin{tikzpicture}[scale=0.65]
    \scriptsize
    \begin{scope}[xscale=-1]
    \def \n {5} \def \radius {3cm} \def \raz {1.4cm}
    \foreach \s in {1,...,\n} { 
    \draw[dashed, >=latex] ({360/\n * (\s - 1)}:\radius)      arc ({360/\n * (\s - 1)}:{360/\n * (\s)}:\radius);   
    \draw[dotted, >=latex] ({360/\n * (\s - 2)+90}:\radius)      -- ({360/\n * (\s)+90}:\radius); 
    } 
    
    \node[state, fill=white] (1) at ({360/\n * (1 - 1)+90}:\radius) {};  
    \node[state, fill=black] (1a) at ({360/\n * (1 - 1)+90}:-\radius/2.618) {}; 
    
    \node[state, fill=blue] (2) at ({360/\n * (2 - 1)+90}:\radius){};   
    \node[state, fill=black] (2a) at ({360/\n * (2 - 1)+90}:-\radius/2.618) {}; 
    
    \node[state, fill=red] (3) at ({360/\n * (3 - 1)+90}:\radius) {};  
    \node[state, fill=black] (3a) at ({360/\n * (3 - 1)+90}:-\radius/2.618) {}; 
    
    \node[state, fill=yellow] (4) at ({360/\n * (4 - 1)+90}:\radius) {}; 
    \node[state, fill=black] (4a) at ({360/\n * (4 - 1)+90}:-\radius/2.618) {}; 
    
    \node[state, fill=green] (5) at ({360/\n * (5 - 1)+90}:\radius) {};   
    \node[state, fill=black] (5a) at ({360/\n * (5 - 1)+90}:-\radius/2.618) {}; 
    
    
    \draw[>=triangle 45, line width=1.5pt, shorten >=1.5ex, ->] (1) -- (4a);
    \end{scope}
    \end{tikzpicture}
    %\caption{Simple move}
    
    You can move \textbf{as far as you want.}
    
    But:\textbf{ you cannot jump!}
     
\newpage
    
   
    \skipper
    \begin{tikzpicture}[scale=0.65]
    \scriptsize
    \begin{scope}[xscale=-1]
    \def \n {5} \def \radius {3cm} \def \raz {1.4cm}
    \foreach \s in {1,...,\n} { 
    \draw[dashed, >=latex] ({360/\n * (\s - 1)}:\radius)      arc ({360/\n * (\s - 1)}:{360/\n * (\s)}:\radius);   
    \draw[dotted, >=latex] ({360/\n * (\s - 2)+90}:\radius)      -- ({360/\n * (\s)+90}:\radius); 
    } 
    
    \node[state, fill=white] (1) at ({360/\n * (1 - 1)+90}:\radius) {};  
    \node[state, fill=white] (1a) at ({360/\n * (1 - 1)+90}:-\radius/2.618) {}; 
    
    \node[state, fill=blue] (2) at ({360/\n * (2 - 1)+90}:\radius){};   
    \node[state, fill=blue] (2a) at ({360/\n * (2 - 1)+90}:-\radius/2.618) {}; 
    
    \node[state, fill=red] (3) at ({360/\n * (3 - 1)+90}:\radius) {};  
    \node[state, fill=red] (3a) at ({360/\n * (3 - 1)+90}:-\radius/2.618) {}; 
    
    \node[state, fill=yellow] (4) at ({360/\n * (4 - 1)+90}:\radius) {}; 
    \node[state, fill=black] (4a) at ({360/\n * (4 - 1)+90}:-\radius/2.618) {}; 
    
    \node[state, fill=green] (5) at ({360/\n * (5 - 1)+90}:\radius) {};   
    \node[state, fill=green] (5a) at ({360/\n * (5 - 1)+90}:-\radius/2.618) {}; 
    
    
    \draw[>=triangle 45, line width=1.5pt, ->] (1) -- (4a) node[midway,right]{$hit$};
    
    \draw[dashed, >=triangle 45, line width=1.5pt, ->] (4a) -- (-2.8,-1) node[midway,right]{$replace$};
    
    \end{scope}
    \end{tikzpicture}
    %\caption{Replace a block}
    
    You can \textbf{hit a black block.} 
    
    You then \textbf{replace} it on another empty space.
    
    


  
    \skipper
    \begin{tikzpicture}[scale=0.65]
    \scriptsize
    \begin{scope}[xscale=-1]
    \def \n {5} \def \radius {3cm} \def \raz {1.4cm}
    \foreach \s in {1,...,\n} { 
    \draw[dashed, >=latex] ({360/\n * (\s - 1)}:\radius)      arc ({360/\n * (\s - 1)}:{360/\n * (\s)}:\radius);   
    \draw[dotted, >=latex] ({360/\n * (\s - 2)+90}:\radius)      -- ({360/\n * (\s)+90}:\radius); 
    } 
    %\draw[dashed] ({360/\n * (1 - 1)+95}:\radius)      arc ({360/\n * (1 - 1)+95}:{360/\n * (1)+85}:\radius)  node[midway, rotate=-18] {$\times$};
    \draw[>=triangle 45, style=double, line width=0.5pt, <->] ({360/\n * (1-1)+95}:\radius)  arc ({360/\n * (1-1)+95}:{360/\n * (2-1)+85}:\radius) node[midway,right]{$swap$};
    
    %\draw[line width=1.5pt, -] ({360/\n * (1)+90}:\radius)      arc ({360/\n * (1)+90}:{360/\n * (2)+90}:\radius) node[midway,below]{};
    
    \node[state, fill=white] (1) at ({360/\n * (1 - 1)+90}:\radius) {};  
    \node[state, fill=white] (1a) at ({360/\n * (1 - 1)+90}:-\radius/2.618) {}; 
    
    \node[state, fill=blue] (2) at ({360/\n * (2 - 1)+90}:\radius){};   
    \node[state, fill=blue] (2a) at ({360/\n * (2 - 1)+90}:-\radius/2.618) {}; 
    
    \node[state, fill=red] (3) at ({360/\n * (3 - 1)+90}:\radius) {};  
    \node[state, fill=red] (3a) at ({360/\n * (3 - 1)+90}:-\radius/2.618) {}; 
    
    \node[state, fill=yellow] (4) at ({360/\n * (4 - 1)+90}:\radius) {}; 
    \node[state, fill=yellow] (4a) at ({360/\n * (4 - 1)+90}:-\radius/2.618) {}; 
    
    \node[state, fill=green] (5) at ({360/\n * (5 - 1)+90}:\radius) {};   
    \node[state, fill=green] (5a) at ({360/\n * (5 - 1)+90}:-\radius/2.618) {}; 
    
    \end{scope}
    \end{tikzpicture}
    %\caption{Swap two pieces}
    
    You can \textbf{swap} two neighbouring pieces \\ (at least one of which must be yours). 
    
    Of course the way must be free!
    
            \skipper
    
    
    \begin{tikzpicture}[scale=0.65]
    \scriptsize
    \begin{scope}[xscale=-1]
    \def \n {5} \def \radius {3cm} \def \raz {1.4cm}
    \foreach \s in {1,...,\n} { 
    \draw[dashed, >=latex] ({360/\n * (\s - 1)}:\radius)      arc ({360/\n * (\s - 1)}:{360/\n * (\s)}:\radius);   
    \draw[dotted, >=latex] ({360/\n * (\s - 2)+90}:\radius)      -- ({360/\n * (\s)+90}:\radius); 
    } 
    
    \node[state, fill=white] (1) at ({360/\n * (1 - 1)+90}:\radius) {};  
    \node[state, fill=white] (1a) at ({360/\n * (1 - 1)+90}:-\radius/2.618) {}; 
    
    \node[state, fill=blue] (2) at ({360/\n * (2 - 1)+90}:\radius){};   
    \node[state, fill=blue] (2a) at ({360/\n * (2 - 1)+90}:-\radius/2.618) {}; 
    
    \node[state, fill=red] (3) at ({360/\n * (3 - 1)+90}:\radius) {};  
    \node[state, fill=red] (3a) at ({360/\n * (3 - 1)+90}:-\radius/2.618) {}; 
    
    \node[state, fill=yellow] (4) at ({360/\n * (4 - 1)+90}:\radius) {}; 
    \node[state, fill=yellow] (4a) at ({360/\n * (4 - 1)+90}:-\radius/2.618) {}; 
    
    \node[state, fill=green] (5) at ({360/\n * (5 - 1)+90}:\radius) {};   
    \node[state, fill=green] (5a) at ({360/\n * (5 - 1)+90}:-\radius/2.618) {}; 
    
    
    \draw[dashed, >=triangle 45, line width=1.5pt, ->] (1) -- (1a);
    
    \draw[>=triangle 45, line width=1.5pt, ->] (1a) -- (1,-1) node[left]{out};
    
    \draw[>=triangle 45, line width=1.5pt, ->] (-0.4,0) -- (-2.4,1.7) node[right]{grey};
    
    
    \end{scope}
    \end{tikzpicture}
    %\caption{Move out, set a new gray block}
    
    When you \textbf{reach a goal,} your piece moves \textbf{out. }
    
    For this you \textbf{place a grey block} anywhere.
    
    Grey blocks are \textbf{one-time blocks.}
    
    
    

    %%%% TURN CORNERS
  
    \skipper
    \begin{tikzpicture}[scale=0.65]
    \scriptsize
    \begin{scope}[xscale=-1]
    \def \n {5} \def \radius {3cm} \def \raz {1.4cm}
    \foreach \s in {1,...,\n} { 
    \draw[dashed, >=latex] ({360/\n * (\s - 1)}:\radius)      arc ({360/\n * (\s - 1)}:{360/\n * (\s)}:\radius);   
    \draw[dotted, >=latex] ({360/\n * (\s - 2)+90}:\radius)      -- ({360/\n * (\s)+90}:\radius); 
    } 
    
    \node[state, fill=white] (1) at ({360/\n * (1 - 1)+90}:\radius) {};  
    \node[state, fill=white] (1a) at ({360/\n * (1 - 1)+90}:-\radius/2.618) {}; 
    
    \node[state, fill=black] (2) at ({360/\n * (2 - 1)+90}:\radius){};   
    \node[state, fill=black] (2a) at ({360/\n * (2 - 1)+90}:-\radius/2.618) {}; 
    
    \node[state, fill=black] (3) at ({360/\n * (3 - 1)+90}:\radius) {};  
    \node[state, fill=black] (3a) at ({360/\n * (3 - 1)+90}:-\radius/2.618) {}; 
    
    \node[state, fill=black] (4) at ({360/\n * (4 - 1)+90}:\radius) {}; 
    \node[state, fill=yellow] (4a) at ({360/\n * (4 - 1)+90}:-\radius/2.618) {}; 
    
    \node[state, fill=black] (5) at ({360/\n * (5 - 1)+90}:\radius) {};   
    \node[state, fill=green] (5a) at ({360/\n * (5 - 1)+90}:-\radius/2.618) {}; 
    
    
    \draw[>=triangle 45, line width=1.5pt, -] (1) -- (4a) node[midway,right]{};
    \draw[>=triangle 45, line width=1.5pt, -] (4a) -- (5a) node[midway,right]{$turn$};
    \draw[>=triangle 45, line width=1.5pt, ->] (5a) -- (1a) node[midway,left]{};
    
    \end{scope}
    \end{tikzpicture}
    
    %\caption{Turn corners on free paths}
    
    \textbf{Turn} at free corners without stopping. 

    Ways can be long!
    
    

  

    \skipper

\raggedright

\vspace{5ex}

Edge cases:

    \begin{enumerate}
        \item When moving to a corner with multiple pieces, swap with one of them.
        \item You can't try the same twice.
        \item When you get to set both a grey and a black block say `Abracadabra'.
        \item You cannot move out when it is not your turn.
        \item When it is your turn and one of your pieces was brought to its goal by someone else, then you must move that piece out, gaining a grey block. You do not have another move.
        \item When you need more grey blocks than there are re-position one.
    \end{enumerate}

\hrulefill

%\skipper

%\vspace{5ex}


%Please visit \website ! 


%\end{multicols}