
\renewcommand{\headline}{Penta$\cdot$permeinan - Bahasa Indonesia}
\renewcommand{\tocent}{Bahasa Indonesia}
\renewcommand{\translator}{Iqbal Akbar, Sabrina Budiono}
\renewcommand{\general}{
    \lettrine[lines=3]{S}{ebuah} permainan yang dapat dijelaskan dalam satu menit tetapi tetap menarik selama bertahun-tahun. 
    
    Dua pemain dapat bermain hanya dalam waktu 20 menit. Tiga atau empat pemain mungkin membutuhkan waktu hingga 90 menit.  
    
    Permainan ini tidak membutuhkan dadu dan dapat dimainkan oleh semua orang mulai dari usia 5 tahun. 
    %Sebuah permainan oleh Jan \noun{Suchanek.}
    
    Di dalam kotak: 4$\times$5 bidak berwarna, 5 kotak hitam dan
    5 kotak abu-abu, papan permainan.
}

\renewcommand{\choosext}{Pilih Bidak Anda}
\renewcommand{\choosex}{
    Setiap pemain memiliki 5 bidak. Di dalam kotak terdapat satu tim bidak untuk setiap pemain. 
    %Satu tim memiliki bidak dengan rambut perak, satu dengan rambut hitam, satu dengan rambut emas, dan lainnya botak.
    
    %Para pemain harus bermain tim yang mirip dengan mereka.
    
    Anda memiliki bidak berwarna biru, merah, putih, hijau, dan kuning.
    
    Para pemain memulai dari kelima sudut di papan permainan sesuai dengan warnanya.
    
    Mereka harus mencapai tempat permberhentian di tengah papan permainan, dimana mereka keluar.
}

\renewcommand{\setupt}{Tata Permainan}
\renewcommand{\setup}{
    Taruh bidak anda di sudut besar di sisi luar lingkaran, sesuai dengan warna bidak: bidak putih anda di sudut putih, bidak biru anda
    di sudut biru, dan seterusnya.
    
    Taruh blok hitam di lima persimpangan di tengah papan permainan.
    
    Taruh blok abu-abu di tengah untuk tahap selanjutnya. 
}

\renewcommand{\objectivet}{Objektif}
\renewcommand{\objective}{
    Bidak putih harus dimainkan menuju persimpangan berwarna putih di tengah, bidak
    biru menuju persimpangan berwarna biru dan seterusnya. Tujuan akhirnya selalu
    titik akhir di tengah, berlawanan dengan titik awal. 
    
    Pemain pertama yang berhasil memindahkan \emph{three} bidak ke tujuan akhir adalah
    pemenangnya.
}

\renewcommand{\rulest}{Peraturan}
\renewcommand{\rules}{
    Pindahkan bidak anda ke bintang atau lingkaran ke berbagai arah dan jarak yang diinginkan. 
    
    Anda bisa berbelok di sudut bebas atau persimpangan tanpa harus berhenti.
    
    Jangan pernah lompat\textemdash melewati block ataupun bidak. 
    
    \myskip
    
    Tetapi anda diperkenankan berhenti \emph{di sana} jika tempat itu sudah terisi:
    
    \myskip
    
    Jika anda mengalahkan blok hitam, tempatkan blok tersebut ke tempat bebas sesuai pilihan anda.
    
    Jika anda bergerak ke tempat yang terisi bidak lain, anda dapat bertukar posisi dengan bidak itu. 
     
    \myskip
    
    Anda diperkenankan menukar posisi dua bidak anda.  
    
    Jika anda bergerak ke tempat dengan bidak lebih dari 1, pilih satu untuk ditukar. 
     
    \myskip 
     
    Jangan membuat gerakan yang sama dua kali setelah pergantian dadakan.   
    
    \myskip 
    
    Sebuah bidak yang mencapai titik akhir dipindahkan dan ditaruh di tengah papan permainan. Kemudian ambil satu blok abu-abu dan taruh di tempat bebas pilihan anda. 
    
    Jika anda mengalahkan blok abu-abu, ambil kembali dari papan permainan.
    
    \myskip
    
    Jika pemain lain telah memindahkan anda ke tujuan anda, anda \emph{harus} berpindah ketika giliran anda.
    
    Babak terakhir dimainkan sampai akhir.
}
