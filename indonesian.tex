\renewcommand{\headline}{Penta$\cdot$permeinan - Bahasa Indonesia}
\renewcommand{\tocent}{Bahasa Indonesia}
\renewcommand{\translator}{Iqbal Akbar}
\renewcommand{\general}{
Permainan yang bisa dijelaskan dalam hitungan menit tetapi meninggalkan kesan
mendalam selama tahunan ke depan. 

Dua pemain dapat bermain dalam waktu 20 menit. Tiga atau empat pemain
membutuhkan waktu 90 menit.  

Permainan ini tidak membutuhkah dadu dan bisa dimainkan oleh semua orang 
mulai dari usia 5 tahun. Sebuah permainan oleh Jan \noun{Suchanek.}

Di dalam kotak: 4$\times$5 gambar lukisan tangan, 5 hitam dan
5 abu-abu kotak, papan permainan.
}

\renewcommand{\choosext}{Pilih Bidak Anda}
\renewcommand{\choosex}{
Setiap pemain memiliki 5 bidak. Di dalam kotak terdapat satu tim bidak untuk 
tiap pemain. 
%Satu tim memiliki bidak dengan rambut perak, satu dengan rambut hitam, satu dengan rambut emas, dan lainnya botak.

%Para pemain harus bermain tim yang mirip dengan mereka.

Anda memiliki bidak berwarna biru, merah, putih, hijau, dan kuning.

Para pemain memulai dari kelima sudut di papan permainan sesuai dengan warnanya.

Mereka harus mencapai titik awal di tengah papan permainan.
}

\renewcommand{\setupt}{Tata Permainan}
\renewcommand{\setup}{
Taruh bidak anda di sudut besar di sisi luar yang sesuai dengan warna 
bidak: bidak putih anda di sudut putih, bidak biru anda
di sudut biru, dan seterusnya.

Taruh blok hitam di lima persimpangan di tengah papan permainan.

Taruh blok abu-abu di tengah untuk tahap selanjutnya. 
}

\renewcommand{\objectivet}{Objektif}
\renewcommand{\objective}{
Bidak putih harus dimainkan menuju persimpangan berwarna putih di tengah, bidak
biru menuju persimpangan berwarna biru dan seterusnya. Tujuan akhirnya selalu
titik akhir di tengah, berlawanan dengan titik awal. 

Pemain pertama yang berhasil memindahkan \emph{three} bidak ke tujuan akhir adalah
pemenang.
}

\renewcommand{\rulest}{Peraturan}
\renewcommand{\rules}{
Pindahkan bidak anda ke bintang atau tengah dengan tujuan yang diinginkan. 

Anda bisa berputar di sudut bebas atau persimpangan tanpa harus berhenti.

Jangan pernah lompat\textemdash block ataupun bidak. 

\myskip

Tetapi anda diperkenankan berhenti \emph{di sana} jika penuh:

\myskip

Jika anda mengalahkan blok hitam, tempatkan blok tersebut ke stop bebas.

Jika anda bergerak ke stop yang terisi bidak lain, gantikan posisinya. 
 
\myskip

Anda diperkenankan menganti posisi dengan dua bidak anda.  

Jika anda bergerak ke stop dengan bidak yang banyak, pilihlah satu untuk digantikan. 
 
\myskip 
 
Jangan membuat gerakan yang sama dua kali setelah pergantian dadakan.   

\myskip 

Sebuah bidak yang mencapai titik akhir dipindahkan dan ditaruh di tengah papan permainan. Kemudian ambil satu blok abu-abu dan taruh di stop bebas pilihan anda. 

Jika anda mengalahkan blok abu-abu, ambillah dari papan permainan kembali.

\myskip

Siapa pun yang menggerakkan \emph{tiga }bidaknya disematkan sebagai pemenang.

\myskip

%Diskualifikasi jika mengobrol banyak.  
}
