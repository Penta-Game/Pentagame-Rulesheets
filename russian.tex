\renewcommand{\headline}{Пента$\cdot$игра - Русский}
\renewcommand{\tocent}{Русский}
\renewcommand{\translator}{alg}
\renewcommand{\general}{
    \lettrine[lines=3]{П}{ростая} и интересная игра. 
    Игра в среднем длится от 20 до 90 минут, в зависимости от количества игроков. 
    В коробке: 4$\times$5 разноцветных фигурок, 5 черных и
    5 серых фигурок, игровое поле.
}

\renewcommand{\choosext}{Выбор фигурок}
\renewcommand{\choosex}{
    Каждый игрок получает пять разноцветных фигурок одного типа.
    
    Фигурки стартуют в вершинах пентаграммы.
    
    Фигурки хотят достигнуть клетки  соответствующего цвета в середине поля.
}
\renewcommand{\setupt}{Подготовка}
\renewcommand{\setup}{
    Расставьте фигурки в вершинах пентаграммы в соответствии с их цветом.
    
    Расставьте черные фигурки на цветные клетки на пересечениях линий.
    
    Поставьте серые фигурки в середину поля \hbox to 0.8em{--\hss--}  они понадобятся позже.
}

\renewcommand{\objectivet}{Цель}
\renewcommand{\objective}{
    Выигрывает первый, кто переместит три фигурки в соответствующие их цвету клетки на пересечениях линий пентаграммы.
}

\renewcommand{\rulest}{Правила}
\renewcommand{\rules}{
    В свой ход можно переместить любую свою фигурку на любое расстояние вдоль линий пентаграммы или кольца.
    
    На развилках можно поворачивать без остановки.
    
    Нельзя перепрыгивать через другие фигурки.
    
    \myskip
    
    Если остановиться \emph{на} занятой клетке:
    
    \myskip
    
    Если в клетке черная фигурка \hbox to 0.8em{--\hss--} переместите ее в любую свободную клетку.
    
    Если в клетке фигурка, принадлежащая игроку \hbox to 0.8em{--\hss--} поставьте ее в клетку, в которой начинался ход.
     
    \myskip
    
    Если остановиться на клетке с несколькими фигурками \hbox to 0.8em{--\hss--} заместите одну из них.
     
    \myskip 
     
    Нельзя повторять один и тот же ход дважды.
    
    \myskip 
    
    Фигурка, достигнув назначенной клетки, удаляется с поля.
    
    Если фигурка достигла цели в результате хода другого игрока, она должна покинуть поле при своем следующем ходе.
    
    Игрок, которому принадлежит фигурка, взамен получает серую фигурку, которую можно поставить на любую свободную клетку.

    При замещении, серая фигурка удаляется с поля.%
 
    Если другой игрок переместил вас к вашей цели, вы должны выйти, как только наступит ваш ход.
    
    Последний раунд разыгрывается до конца.
}