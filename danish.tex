\selectlanguage{danish}
\renewcommand{\headline}{Penta$\cdot$spil - Dansk}
\renewcommand{\tocent}{Dansk}
\renewcommand{\translator}{Thomas Dorf Nielsen}
\renewcommand{\general}{
    \lettrine[lines=3]{E}{t} spil, som man forstår på et minut, men som stadig fascinerer efter flere år. 
    Er man to, varer et spil kun 20 minutter. For tre-fire spillere varer et spil 40-90 minutter.
    Spilles uden terninger. Egnet til spillere fra 5 år. 
    Indhold: 4$\times$5 farvede figurer, 5 sorte og 5 grå kuber og spillepladen.
}

\renewcommand{\choosext}{Figurer}
\renewcommand{\choosex}{
    Hver spiller vælger én af de fire figurer og tager alle fem farvede udgaver af figuren: Blå, rød, hvid, grøn og gul.
    Figurerne begynder på den yderste ring på det felt, der har samme farve som figuren.
    Figurerne skal ende på den inderste ring på det felt, der har samme farve som figuren.
}

\renewcommand{\setupt}{Opsætning}
\renewcommand{\setup}{
    Stil figurerne på de fem farvede felter på den yderste ring, sorteret efter farve: De blå figurer skal stå på det blå felt, de røde figurer på det røde felt osv.
    
    Stil en sort kube på hvert af de fem farvede felter på den inderste ring.
    
    De grå kuber stilles på den tomme plads midt på spillepladen.
}

\renewcommand{\objectivet}{Spillets mål}
\renewcommand{\objective}{
    Hver figur skal forsøges flyttet til det felt på den inderste ring, der har den samme farve som figuren. Hvide figurer skal flyttes til det hvide felt på den inderste ring, grønne figurer til det grønne felt osv.
    Målet ligger altid på den inderste ring, længst væk fra startfeltet.
    
    Vinderen er den, der først får \emph{tre }af sine figurer i mål.
}

\renewcommand{\rulest}{Regler}
\renewcommand{\rules}{
    Når det er din tur, må du flytte én af dine figurer lang linjerne.
    Man må flytte figuren så langt, man kan, i alle retninger.
    
    Man må dreje i et kryds og fortsætte, hvis krydset er ledigt.
    
    Man må \emph{aldrig }springe over andre brikker, hverken kuber eller andre figurer!
    
    \myskip
    
    Man må \emph{gerne }flytte hen på et felt, der allerede er optaget af en anden brik. Denne anden brik bliver derved slået væk og skal placeres et nyt sted:
    
    \myskip
    
    Slås en sort kube væk, skal den placeres på et frit felt efter eget valg.
    
    Slås en anden spillerbrik væk, skal den placeres på det felt, hvor den først flyttede brik kom fra. De to spillerbrikker bytter altså plads. På denne måde kan man bytte rundt på to af sine egne figurer.
    
    Flyttes til et felt, hvor der står flere figurer, skal man bytte plads med \emph{én }af disse figurer.
    
    Slås en grå kube væk, stilles den tilbage i det tomme felt på midten af spillepladen.
     
    \myskip 
     
    Man må ikke foretage det samme træk to gange lige efter hinanden.
    
    \myskip 
    
    Når en figur når sit mål, flyttes den til midten af spillepladen. Spilleren får en grå kube, som må placeres på et ledigt felt på spillepladen.
    
    De grå kuber sættes tilbage på midten af spillepladen, når de slås væk.
    
    \myskip 
    
    Hvis én af dine figurer bliver sendt i mål af en anden spiller, så \emph{skal }du bruge din næste tur på at flytte denne figur ind på midten af spillepladen.
    
    \myskip
    
    Hvis en anden spiller flytter dig til målet, skal det slettes, når det er din tur.
    
    \myskip
    
    Den sidste runde spilles til slutningen.
}