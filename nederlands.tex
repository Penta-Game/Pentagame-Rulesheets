%\selectlanguage{nederlands}
\renewcommand{\headline}{Penta$\cdot$spel - Nederlands}
\renewcommand{\tocent}{Nederlands}
\renewcommand{\translator}{Daniel Franke}
\renewcommand{\general}{
    \lettrine[lines=3]{E}{en} spel dat men in een minuut snapt, maar dat jarenlang blijft fascineren.
    Met twee spelers duurt een spel 20 minuten. Met drie of vier spelers duurt het 40 tot 90 minuten.

    Er zijn geen dobbelstenen. Geschikt voor iedereen vanaf 5 Jaar. Ontworpen door Jan \noun{Suchanek.}

    In de doos: 4$\times$5 gekleurde figuren, 5 zwarte en 5 grijse blokkades en het spelbord.
}

\renewcommand{\choosext}{Figuren}
\renewcommand{\choosex}{
    Elke speler heeft een team van vijf figuren van een vorm naar keuze.

    In elk team zit een rode, een blauwe, een witte, een groene en een gele figuur.

    Ze beginnen in de hoeken van de eigen kleur, en eindigen in de kruizingen van de eigen kleur, waar ze uit gaan.
}

\renewcommand{\setupt}{Opbouw}
\renewcommand{\setup}{
    Zet je figuren op de hoeken van dezelfde kleur, het witte figuur op de witte hoek, de gele figuur op de gele hoek, enz.
    
    Zet de zwarte blokkades op de gekleurde kruizingen in het midden.

    Zet de grijze blokkades voor nu in het midden.
}

\renewcommand{\objectivet}{Doel van het spel}
\renewcommand{\objective}{
    Witte figuren moeten naar de witte kruizing, gele figuren naar de gele kruizing, enz.
    Het doel is altijd de kruizing van dezelfde kleur tegenover het beginpunt.

    Wie het eerst \emph{drie} figuren naar hen doel krijgt, wint.
}

\renewcommand{\rulest}{Spelregels}
\renewcommand{\rules}{
    Zet een van je figuren op de ster of ring in een richting, zo ver als je wilt.
    
    Je kan bij elke hoek of kruizing van richting veranderen zonder de zet te beeindigen.

    Je kan nooit overspringen, niet over blokkades, en niet over andere figuren.

    \myskip

    Maar je mag de figuur wel \emph{tot op} een bezet veld zetten en slaan:

    \myskip

    Als je een zwarte blokkade slaat, mag je de blokkade op een ander vrij veld zetten. 

    Als je een ander figuur slaat, ruil je van positie met dat figuur.

    \myskip

    Je kan de posities van twee van je eigen figuren zo omruilen.

    Wanneer je naar een veld met meerdere figuren zet, moet je \emph{één} van de figuren slaan.

    \myskip 
 
    Je mag niet twee keer achter elkaar dezelfde zet maken.

    \myskip 

    Een figuur dat zijn doel heeft berreikt is uit het spel. Deze wordt in het midden van het bord gezet. Je kan nu een van de grijze blokkades op een vrij veld van het bord zetten.

    Wanneer een van deze grijze blokkades wordt geslagen, wordt deze weer uit het spel gezet.

    \myskip 

    Wanneer een andere speler je figuur op zijn doel heeft gezet, \emph{moet} je deze uit zetten met je volgende beurt.

    \myskip

    De laatste ronde wordt tot het einde gespeeld.
    % Wie het eerst     \emph{drie} van zijn figuren uit heeft, wint.
}
