\selectlanguage{swedish}
\renewcommand{\headline}{Penta$\cdot$spel - Svenska}
\renewcommand{\tocent}{Svenska}
\renewcommand{\translator}{Excds}
\renewcommand{\general}{
\lettrine[lines=3]{E}{tt} spel som kan förklaras på under en minut, men som likväl fortsätter fascinera i åratal. 
Ett tvåpersonersparti tar bara 20 minuter. Med tre eller fyra spelare upp till
90 minuter.

Det spelas utan tärning och kan spelas från fem års ålder. Spelet är skapat av
Jan \noun{Suchanek.}

Innehåll i kartongen: 4$\times$5 färgade pjäser, 5 svarta och 5 grå
blockerarpjäser, 1 spelbräde.
}

\renewcommand{\choosext}{Val av spelpjäser}
\renewcommand{\choosex}{
Varje spelare har fem spelpjäser. Det finns en grupp pjäser för varje spelare i
lådan.

Du har en blå, en röd, en vit, en grön och en gul pjäs.

Varje pjäs har som startposition fältet med matchande färg längs spelbrädets
ytterkant.

Varje pjäs vill gå till fältet med matchande färg i spelbrädets mitt.
}
\renewcommand{\setupt}{Spelstart}
\renewcommand{\setup}{
Ställ pjäserna i de stora färgade cirklarna runt spelbrädets kant i matchande
färger:
Din vita pjäs i det vita fältet, den blå i det blå fältet osv.

Ställ en av de svarta blockerarpjäserna i vardera färgat fält i mitten av
spelbrädet.

Ställ de grå pjäserna i mitten av spelbrädet för senare användning.
}

\renewcommand{\objectivet}{Spelets mål}
\renewcommand{\objective}{
Vita figurer vill röra sig till det vita fältet i minnen, blå till det blå
fältet osv. Målet är alltid det färgade fältet i motsatt riktning på pentagonen
sett från ursprungspositionen.

För att vinna så ska du vara den första spelare som har fått \emph{tre} pjäser
till fälten med deras motsvarande färger på spelbrädet.
}

\renewcommand{\rulest}{Regler}
\renewcommand{\rules}{
Du kan flytta dina pjäser i godtycklig riktning runt spelbrädet så långt som du
önskar.

Du kan flytta förbi varje öppet hörn eller fri korsning utan att stanna.

Du kan inte hoppa över en blockerarpjäs eller andra pjäser.

\myskip

Du kan flytta en pjäs \emph{till} en position som redan är ockuperad.

\myskip

Om du isåfall flyttar till en position med en svart blockerarpjäs, kan du
flytta den till godtycklig ledig plats på brädet.

Om du flyttar till en position där det står en motspelares pjäs, byt plats på
pjäserna.

\myskip

Du kan byta plats på två av dina egna spelpjäser.

Om du flyttar till en position med flera pjäser, välj en av pjäserna att byta
plats med.

\myskip

Utför inte samma drag två gånger efter varandra.

\myskip

En pjäs som har nått sitt mål plockas bort från brädet. Ställ pjäsen i mitten
av spelbrädet, varpå du tar en av de grå blockerarpjäserna och ställer på
valfri ledig position.

Om du flyttar till en position där det står en grå blockerarpjäs, flytta då
tillbaka den grå pjäsen till mitten av spelbrädet.

\myskip

Om en annan spelare flyttar dig till målet måste den raderas när det är din tur.

\myskip

Den sista omgången spelas till slutet.
}