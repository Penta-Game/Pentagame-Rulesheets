\renewcommand{\headline}{Penta$\cdot$game – 繁體中文}

\renewcommand{\tocent}{Traditional Chinese}
\renewcommand{\translator}{藍文婷}

\renewcommand{\general}{一個只需數分鐘解釋規則卻可令人著迷多年的遊戲。

  2位玩家遊戲時間為20分鐘。3或4名玩家則為90分鐘。
遊戲不需要骰子。本遊戲適合5歲以上的玩家。Jan \noun{Suchanek }遊戲設計。 

盒中含有:4組 5個手繪角色,5塊黑色以及5塊灰色木塊以及底板
} 

\renewcommand{\choosext}{選擇你的角色} 
\renewcommand{\choosex}{
每位玩家有五個角色。盒中每個玩家都有一個角色團隊。
%一隊銀髮、一隊黑髮、一隊金髮以及一隊無髮。 
%玩家應扮演與自己相似的團隊。
玩家有藍色、紅色、白色、綠色以及黃色的角色。
角色從底板五個與自身相同顏色的角落開始遊戲。
目標是到達中間的大站點。
}

\renewcommand{\setupt}{設置} 
\renewcommand{\setup}{
把你的角色放在相同顏色的大的角落邊緣:白色角色放在白色角落邊緣,藍色角色則放在藍色角落邊緣,以此類推。
把黑色的木塊放在底板中間五個角落的交會點。
灰色的木塊稍後置於中間。
}

\renewcommand{\objectivet}{目標}
\renewcommand{\objective}{
白色的角色要走到中間的白色交會點,藍色角色要走到藍色交會點,以此類推。目的地為出發點對面,位於中間的的彩色大站點。

第一個讓\emph{``三個''}角色到達目的地的人即為贏家。 
}

\renewcommand{\rulest}{遊戲規則}
\renewcommand{\rules}{
將起始點或邊緣的任何角色移動到任意方向,方向距離可自行決定。
玩家可以轉向任何空的角落或不停留直接穿越。
不能跳過\textemdash 任何木塊或者任何角色。

\myskip

但玩家可以移動 \emph{``至''}被佔據的站點:

\myskip

如果玩家戰勝了黑色木塊,可將黑色木塊放置於任何空站點。
若玩家移動至已經有角色的站點,則需與該角色交換位置。
\myskip

玩家可以與自己的兩個角色交換位置。
若移動到有多個角色的站點,可與一個角色交換位置。
\myskip

不要連續兩次進行相同的移動。
\myskip

請將到達目的地的角色移除,將之移動到底板中央。將一塊灰色木塊隨意擺放至空站點。
若你戰勝灰色木塊,將木塊從底板中拿走。
\myskip

任何人\emph{``三個''}首先將角色從遊戲中移除者即獲得勝利。
\myskip

%交談過多的玩家會失去資格。
}


%%EOF
