\renewcommand{\headline}{Penta$\cdot$game – 繁體中文}

\renewcommand{\tocent}{繁體中文}%
\renewcommand{\translator}{藍文婷}

\renewcommand{\general}{
    一個只需數分鐘解釋規則卻可令人著迷多年的遊戲。
    
    $2$ 位玩家遊戲時間為 $20$ 分鐘。$3$或$4$ 名玩家則為 $90$ 分鐘。
    遊戲不需要骰子。本遊戲適合 $5$ 歲以上的玩家。
    %$Jan$  $Suchanek$  遊戲設計。 
    
    盒中含有:$4$組$5$ 個手繪角色,$5$ 塊黑色以及5塊灰色木塊以及底板
} 

\renewcommand{\choosext}{選擇你的角色} 
\renewcommand{\choosex}{
    每位玩家有五個角色。盒中每個玩家都有一個角色團隊。
    玩家有藍色、紅色、白色、綠色以及黃色的角色。
    角色從底板五個與自身相同顏色的角落開始遊戲。
    目標是到達中間的大站點。
}

\renewcommand{\setupt}{設置} 
\renewcommand{\setup}{
    把你的角色放在相同顏色的大的角落邊緣:白色角色放在白色角落邊緣,藍色角色則放在藍色角落邊緣,以此類推。
    把黑色的木塊放在底板中間五個角落的交會點。
    灰色的木塊稍後置於中間。
}

\renewcommand{\objectivet}{目標}
\renewcommand{\objective}{
    白色的角色要走到中間的白色交會點,藍色角色要走到藍色交會點,以此類推。目的地為出發點對面,位於中間的的彩色大站點。
    
    第一個讓\emph{``三個''}角色到達目的地的人即為贏家。 
}

\renewcommand{\rulest}{遊戲規則}
\renewcommand{\rules}{
    將起始點或邊緣的任何角色移動到任意方向,方向距離可自行決定。
    玩家可以轉向任何空的角落或不停留直接穿越。
    不能跳過\textemdash 任何木塊或者任何角色。
    
    \myskip
    
    但玩家可以移動 \emph{``至''}被佔據的站點:
    
    \myskip
    
    如果玩家戰勝了黑色木塊,可將黑色木塊放置於任何空站點。
    若玩家移動至已經有角色的站點,則需與該角色交換位置。
    \myskip
    
    玩家可以與自己的兩個角色交換位置。
    若移動到有多個角色的站點,可與一個角色交換位置。
    \myskip
    
    不要連續兩次進行相同的移動。
    \myskip
    
    請將到達目的地的角色移除,將之移動到底板中央。將一塊灰色木塊隨意擺放至空站點。
    若你戰勝灰色木塊,將木塊從底板中拿走。
    \myskip
    
    如果其他玩家將您移至目標,則該輪到您時必須將其刪除。
    
    \myskip
    
    最後一輪進行到最後。
}


%%EOF
