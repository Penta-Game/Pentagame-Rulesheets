\begingroup
\footnotesize
\frakfamily
In blosz f{\"u}nf Mi\-nut\-en begreifbar, jedoch jahrelang fas:zinierend. Zu zweit dauert die Partie nur zwanzig Minuten. Drei oder vier Spieler spielen vierzig bis: neunzig Minuten. Es: gibt keine 
W{\"u}rfel. Geeignet f{\"u}r Spieler ab f{\"u}nf Jahren. Ein Spiel von Jan Suchanek. In der Schachtel: 4 mal 5 farbige Figuren, 5 schwarze und 5 graue Bl"ocke und das Spielbrett.
\so{Figuren:}
Jeder Spieler f{\"u}hrt eine Mannschaft von f{\"u}nf Figuren einer Form.
Jeder sucht sich eine Mannschaft aus:.
In jeder solchen Mannschaft ist  eine blaue, eine rote, eine weisze, eine gr{\"u}ne und eine gelbe Figur.
Diese laufen alle von den f{\"u}nf Ecken zu den f{\"u}nf Kreuzungen, wo sie herausziehen.
\so{Aufbau:}
Stellt die Figuren geordnet nach Farben auf die f{\"u}nf Ecken auf dem Kreis: die weiszen auf das weisze Eckfeld, die blauen auf das blaue usw.
Je ein schwarzer Block kommt auf die Kreuzungen in der Mitte.
Die grauen Blöcke bleiben erst mal im Zentrum.
\so{Ziel: }
Weisze Figuren möchten zu der weiszen Kreuzung, blaue zur blauen usw. 
Vom Rand her ist das Ziel immer die Kreuzung gleicher Farbe gegen{\"u}ber.
Wer zuerst \so{drei} seiner Figuren auf ihre jeweiligen Ziele zieht, gewinnt.
\so{Spielregeln: }
Ziehe eine deiner Figuren auf Stern oder Kreis: in beliebiger Richtung beliebig weit.
Du kannst dabei an jeder freien Kreuzung ohne anzuhalten abbiegen.
Du darfst ziehen, aber nicht springen! Weder {\"u}ber Blöcke noch {\"u}ber andere Figuren!
Jedoch kannst du auf ein Feld ziehen, das besetzt ist, und schlagen:
Schlägst du einen schwarzen Block, so setze ihn auf ein beliebiges: freies Feld. 
Schlägst du eine andere Spielfigur, so tauschen deine und diese Figur ihre Positionen.
Auf diese Weise kann man zwei seiner eigenen Figuren vertauschen.
Ziehst du auf ein Feld, auf dem noch mehrere Figuren stehen, muszt du mit \so{einer} von ihnen tauschen.
Man darf nicht denselben Zug zweimal machen. 
Hat eine Figur ihr Ziel erreicht, so zieht sie raus:. Sie kommt ins Zentrum. Daf{\"u}r darf man von dort einen grauen Block ins Brett auf ein freies Feld setzen. 
Schl{\"a}gt man so einen grauen Block, so kommt er wieder raus:.
Wirst du von einem anderen Spieler auf dein Ziel gebracht, so muszt du raus: 
ziehen, sobald du dran bist. Die letzte Runde wird
zu Ende gespielt.
\endgroup
%