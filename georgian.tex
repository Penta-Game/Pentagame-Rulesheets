\renewcommand{\headline}{პენტა $\cdot$ გამა - 
ქართული}
\renewcommand{\tocent}{ქართული}
\renewcommand{\translator}{Irakli Sakvarelidze}
\renewcommand{\general}{
    \lettrine[lines=3,loversize=0.7]{თ}{ამაში,} რომელიც მარტივი და ამავე დროს მიმზიდველია. 
    თამაშის ხანგრძლივობა გრძელდება 20წ.-დან 90წ.-მდე, რაც განისაზღვრება იმის და მიხედვით თუ რამდენი მოთმაშე მონაწილეობს.
    
    ყუთში მოთავსებულია, სათამაშო დაფასთან ერთდ 4$\times$5 სხვადახვა ფერის ფიგურები, 5 შავი და 5
    სერი.
}

\renewcommand{\choosext}{ფიგურების არჩევა}
\renewcommand{\choosex}{
ყოველი მოთმაშე იღებს 5 სხვადასხვა ფერის, მსგავსი ფორმის, ფიგურებს. 

ფიგრურები სტარტს იღებენ დიდ წრეზე განთავსებული 5 ქიმერიდან. 

მათი მოვალეობაა, მიაღწიონ ცენტრალურ უბანს.

}
\renewcommand{\setupt}{მომზადება}
\renewcommand{\setup}{
დაალაგეთ ფიგურები ფერების მიხედვით, წრიულად ქიმებზე. 

შავი ფიგურები დაალაგეთ ცენტრში წრიულად გადამკვეთ ხაზებთან. 

სერი ფიგურები კი მოათავსეთ ცენტრში, ისინი გამოგვადგება შემდგომ.
}

\renewcommand{\objectivet}{მიზანი}
\renewcommand{\objective}{
იგებს ის ვინც პირველი გადაადგილებს სამ ფიგურას ფერების შესაბამის უჯრებში ხაზების გადაკვეთის წერტილებში. 

ამიტომ, იყავი პირველი.
}

\renewcommand{\rulest}{წესები}
\renewcommand{\rules}{

მოთამაშეს შეუძლია მოთამაშის ნებისმიერი ფიგურა გადაადგილოს ნებისმიერ დისტანციაზე წრიულად ან გადაკვეთის წერტილებში. 

სვლა შეგიძლია გააგრძელო იმ შემთხვევაში თუ წინ ფიგურა არ გხვდება. გადახტომა არ შეიძლება. 

თუ შავი ფიგურა შეგხვდა გადაადგილე ნებისმიერ ცარიელ უჯრაში. 

\myskip

თუ გადამკვეტ უჯრაში მოწინააღმდეგის ფიგურაა შეგხვდა დაბრკოლებად, გაუცვალე ადგილი. 

\myskip

თუ მიადეგით ადგილს სადაც რამოდენიმე მეტოქის ფიგურაა თავმოყრილი, ჩაანაცვლე ერთ-ერთი მათგანით. 

არ შეიძლება ერთი და იგივე სვლის ორჯერ გამეორება. 

\myskip

ფიგურა რომელიც მიაღწევს დანიშნულების ადგილს გადის თამაშიდან და მის მაგივრად, მოთამაშე იღებს სერ ფიგურას, რომლის მოთავსებაც შეიძლება ნებისმიერ თავისუფალ ადგილას. 

\myskip
თუ მეორე მოთამაშემ შენი მიზნისკენ გადაგაადგილა, შენც უნდა გადახვიდე, სანამ შენი სვლაა!

\myskip
ბოლო ტური ბოლომდე ითამაშა.
}
