\renewcommand{\headline}{Penta$\cdot$jeux - Fran\c{c}ais}
\renewcommand{\tocent}{Fran\c{c}ais}
\renewcommand{\translator}{Armelle Journaix}
\renewcommand{\general}{ 
\lettrine[lines=3]{U}{n} jeu de société qui s'explique en une minute mais reste fascinant pendant des années. 
À deux, la partie dure environ 20 minutes. 
À trois ou quatre, la partie peut durer jusqu'à 90 minutes.
Aucun dé requis.
Âge : à partir de 5 ans. 
Contenu de la boîte : 4 familles de 5 pions en couleurs, 5 pions noirs, 5 pions gris, le plateau.

}
\renewcommand{\choosext}{Choisissez vos pions}
\renewcommand{\choosex}{
Chaque joueur dispose de cinq pions.
Une famille est composée de cinq pions et il y en a cinq par joueur dans la boîte.

%Il y a la famille aux cheveux argentés, la famille aux cheveux noirs, une famille aux cheveux dorés et une famille chauve.

Chaque équipe est composée de cinq pions : un bleu, un rouge, un blanc, un vert et un jaune.
}

\renewcommand{\setupt}{Installation}
\renewcommand{\setup}{ 
Disposez vos pions sur les cercles extérieurs correspondant aux couleurs des pions : votre pion blanc sur le cercle extérieur blanc, votre pion bleu sur le cercle extérieur bleu, etc.

Disposez les pions noirs sur les cercles intérieurs en couleurs.

Posez les pions gris au centre du plateau de jeu.
}

\renewcommand{\objectivet}{But du jeu}
\renewcommand{\objective}{
Les pions blancs doivent se rendre sur le cercle intérieur blanc, les pions bleus sur le cercle intérieur bleu, etc.

Le but est toujours le cercle intérieur coloré à l'opposé du cercle de départ.

Soyez le premier à déplacer trois de vos cinq pions sur leurs cercles intérieurs colorés respectifs pour gagner.
}

\renewcommand{\rulest}{Règles du jeu}
\renewcommand{\rules}{
Déplacez le pion de votre choix sur le plateau de jeu, où vous voulez et dans n'importe quelle direction.

Vous pouvez utiliser tous les chemins possibles.

Vous ne pouvez pas sauter les cercles occupés par des pions, quelle que soit leur couleur.

\myskip

Mais vous pouvez poser votre pion sur un cercle déjà occupé :

\myskip

Si vous posez votre pion sur un cercle occupé par un pion noir, prenez le pion noir et posez-le ensuite où vous voulez sur un cercle libre.

Si vous déplacez votre pion sur un cercle occupé par un autre pion, vous changez de place avec lui.

\myskip

Vous pouvez changer de place entre deux pions de votre famille.

Si vous déplacez votre pion sur une case occupée par plusieurs autres pions, choisissez-en un et changez de place avec lui.

\myskip

Vous ne pouvez pas effectuer le même déplacement deux fois de suite.

\myskip

Un pion qui atteint son but est retiré du jeu.

Ensuite, prenez l'un des pions gris du centre du jeu et placez-le sur le cercle de votre choix.

Si votre pion se pose à la place d'un pion gris, le pion gris est retiré du jeu.

\myskip

Si un autre joueur vous a déplacé vers votre but, vous devez partir une fois que c'est votre tour.


\myskip

Le dernier tour se joue jusqu'à la fin.
% Le premier joueur qui retire trois de ses pions du plateau gagne la partie.%
}