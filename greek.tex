\renewcommand{\headline}{Πέντε$\cdot$παιχνίδι - Eλληνικά}
\renewcommand{\tocent}{Eλληνικά}
\renewcommand{\translator}{Athanasios Kostopoulos}
\renewcommand{\general}{
\lettrine[lines=3]{E}{να} παιχνιδι που μπορει να εξηγηθει σε ενα λεπτο αλλα παραμενει
συναρπαστικο για χρονια.

Μια παρτιδα δυο παικτων διαρκει μολις 20 λεπτα. Μια παρτιδα
τριων η τεσσαρων παικτων μπορει να διαρκεσει ως 90 λεπτα.

Το παιχνιδι δεν χρησιμοποιει ζαρια και ειναι καταλληλο για ολες τις
ηλικιες απο 5 ετων και πανω. 
%Ο σχεδιαστης του παιχνιδιου ειναι ο Jan \noun{Suchanek.}

Στο κουτι θα βρειτε: 4$\times$5 εγχρωμες φιγουρες, 5 μαυρα and
5 γκριζα τουβλακια, καθως και το ταμπλω παιχνιδιου.
}

\renewcommand{\choosext}{Επιλογη φιγουρων}
\renewcommand{\choosex}{
Καθε παικτης εχει πεντε φιγουρες. Υπαρχει μια ομαδα φιγουρων ανα παικτη στο κουτι.
%Μια ομαδα εχει ασπρα μαλλια, μια ομαδα μαυρα, μια ξανθα και μια ειναι φαλακρη.

%Οι παικτες πρεπει να διαλεξουν την ομαδα με την οποια μοιαζουν περισσοτερο.

Ο καθε παικτης εχει μια μπλε, μια κοκκινη, μια λευκη, μια πρασινη
και μια κιτρινη φιγουρα.

Αρχιζουν στις πεντε γωνιες του ταμπλω, αναλογα με το χρωμα τους.

Ο στοχος ειναι να φτασουν στις μεγαλες στασεις στο κεντρο του ταμπλω.

}
\renewcommand{\setupt}{Στησιμο}
\renewcommand{\setup}{
Τοποθετειστε τις φιγουρες σας στις μεγαλες γωνιες του δακτυλιου, αναλογα με το χρωμα τους:
η λευκη φιγουρα στη λευκη γωνια, η μπλε φιγουρα στην μπλε γωνια, κ.ο.κ.

Τοποθετειστε τα μαυρα τουβλακια στις πεντε διασταυρωσεις στο μεσο του ταμπλω.

Αφηστε τα γκριζα τουβλακια στη μεση του ταμπλω για μετεπειτα χρηση. 
}

\renewcommand{\objectivet}{Στοχος παιχνιδιου}
\renewcommand{\objective}{
Οι λευκες φιγουρες θελουν να πανε στη λευκη διασταυρωση στο κεντρο,
οι μπλε φιγουρες στην μπλε, κ.ο.κ. Ο προορισμος ειναι παντα η μεγαλη,
εγχρωμη σταση στο μεσαιο πενταγωνο, απεναντι απο το σημειο εκκινησης.

Ο πρωτος παικτης που θα τοποθετησει \emph{τρια} κομματια στον προορισμο τους, κερδιζει.
}

\renewcommand{\rulest}{Κανονες παιχνιδιου}
\renewcommand{\rules}{
Μετακινειστε οποιαδηποτε απο τις φιγουρες σας στο αστρο η στο δακτυλιο, σε καθε
κατευθυνση, οσο θελετε.

Μπορειτε να στριψετε σε καθε ελευθερη γωνια η διασταυρωση χωρις να σταματησετε.

Ποτε δεν μπορειτε να πηδηξετε\textemdash ουτε πανω απο τουβλακια, ουτε πανω απο φιγουρες. 

\myskip

Παρολα αυτα, μπορειτε να μετακινηθειτε \emph{σε} μια κατειλλημενη θεση:

\myskip

Αν καταλαβετε θεση με μαυρο τουβλακι, τοποθετηστε το σε μια ελευθερη θεση της επιλογης σας.

Αν μετακινειθειτε σε θεση με μια αλλη φιγουρα, ανταλλαξτε θεσεις.
 
\myskip

Μπορειτε να ανταλλαξετε τις θεσεις δυο απο τις φιγουρες σας.

Αν μετακινειθειτε σε θεση με πολλαπλες φιγουρες, διαλεξτε μια για ανταλλαγη θεσης.
 
\myskip 

Μην κανετε την ιδια κινηση δυο φορες συνεχομενα.   

\myskip 

Μια φιγουρα που φτανει στον προορισμο της αφαιρειται. Τοποθετηστε την στο κεντρο του ταμπλω. Αφου γινει αυτο,
παιρνετε ενα απο τα γκριζα τουβλακια και το τοποθετειτε σε ελευθερη θεση της επιλογης σας.

\myskip

Αν πατε σε θεση με γκριζο τουβλακι, το αφαιρειτε απο το ταμπλω.

\myskip

Εάν μεταφερθείτε στον προορισμό σας από άλλο παίκτη, πρέπει να βγείτε από τη στροφή σας.

\myskip

Ο τελευταίος γύρος παίζεται μέχρι το τέλος.
}
